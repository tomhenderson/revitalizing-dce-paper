% This is "sig-alternate.tex" V2.0 May 2012
% This file should be compiled with V2.5 of "sig-alternate.cls" May 2012
%
% This example file demonstrates the use of the 'sig-alternate.cls'
% V2.5 LaTeX2e document class file. It is for those submitting
% articles to ACM Conference Proceedings WHO DO NOT WISH TO
% STRICTLY ADHERE TO THE SIGS (PUBS-BOARD-ENDORSED) STYLE.
% The 'sig-alternate.cls' file will produce a similar-looking,
% albeit, 'tighter' paper resulting in, invariably, fewer pages.
%
% ----------------------------------------------------------------------------------------------------------------
% This .tex file (and associated .cls V2.5) produces:
%       1) The Permission Statement
%       2) The Conference (location) Info information
%       3) The Copyright Line with ACM data
%       4) NO page numbers
%
% as against the acm_proc_article-sp.cls file which
% DOES NOT produce 1) thru' 3) above.
%
% Using 'sig-alternate.cls' you have control, however, from within
% the source .tex file, over both the CopyrightYear
% (defaulted to 200X) and the ACM Copyright Data
% (defaulted to X-XXXXX-XX-X/XX/XX).
% e.g.
% \CopyrightYear{2007} will cause 2007 to appear in the copyright line.
% \crdata{0-12345-67-8/90/12} will cause 0-12345-67-8/90/12 to appear in the copyright line.
%
% ---------------------------------------------------------------------------------------------------------------
% This .tex source is an example which *does* use
% the .bib file (from which the .bbl file % is produced).
% REMEMBER HOWEVER: After having produced the .bbl file,
% and prior to final submission, you *NEED* to 'insert'
% your .bbl file into your source .tex file so as to provide
% ONE 'self-contained' source file.
%
% ================= IF YOU HAVE QUESTIONS =======================
% Questions regarding the SIGS styles, SIGS policies and
% procedures, Conferences etc. should be sent to
% Adrienne Griscti (griscti@acm.org)
%
% Technical questions _only_ to
% Gerald Murray (murray@hq.acm.org)
% ===============================================================
%
% For tracking purposes - this is V2.0 - May 2012

\documentclass{sig-alternate}
\usepackage{array}
\usepackage{tikz}
\usepackage{color}
\usepackage{hyperref}
\usepackage{multicol}
\usepackage[export]{adjustbox}
%\usepackage{xcolor}
\usepackage{listings}
\newcommand{\subparagraph}{}
%\usepackage{titlesec}

\lstdefinestyle{CStyle}{
    basicstyle=\footnotesize,
    breakatwhitespace=false,         
    breaklines=true,                 
    captionpos=b,                    
    keepspaces=true,                 
    numbers=left,                    
    numbersep=5pt,                  
    showspaces=false,                
    showstringspaces=false,
    showtabs=false,                  
    tabsize=2,
    language=C
}

\setcounter{secnumdepth}{4}

%\titleformat{\paragraph}
%{\normalfont\normalsize}{\theparagraph}{1em}{}

\begin{document}
%
% --- Author Metadata here ---
%\conferenceinfo{WOODSTOCK}{'97 El Paso, Texas USA}
%\CopyrightYear{2007} % Allows default copyright year (20XX) to be over-ridden - IF NEED BE.
%\crdata{0-12345-67-8/90/01}  % Allows default copyright data (0-89791-88-6/97/05) to be over-ridden - IF NEED BE.
% --- End of Author Metadata ---

\title{Revitalizing ns-3's Direct Code Execution}

%\titlenote{A full version of this paper is available as
%\textit{Author's Guide to Preparing ACM SIG Proceedings Using
%\LaTeX$2_\epsilon$\ and BibTeX} at
%\texttt{www.acm.org/eaddress.htm}}}
%
% You need the command \numberofauthors to handle the 'placement
% and alignment' of the authors beneath the title.
%
% For aesthetic reasons, we recommend 'three authors at a time'
% i.e. three 'name/affiliation blocks' be placed beneath the title.
%
% NOTE: You are NOT restricted in how many 'rows' of
% "name/affiliations" may appear. We just ask that you restrict
% the number of 'columns' to three.
%
% Because of the available 'opening page real-estate'
% we ask you to refrain from putting more than six authors
% (two rows with three columns) beneath the article title.
% More than six makes the first-page appear very cluttered indeed.
%
% Use the \alignauthor commands to handle the names
% and affiliations for an 'aesthetic maximum' of six authors.
% Add names, affiliations, addresses for
% the seventh etc. author(s) as the argument for the
% \additionalauthors command.
% These 'additional authors' will be output/set for you
% without further effort on your part as the last section in
% the body of your article BEFORE References or any Appendices.

\numberofauthors{2} %  in this sample file, there are a *total*
% of EIGHT authors. SIX appear on the 'first-page' (for formatting
% reasons) and the remaining two appear in the \additionalauthors section.
%
\author{
% You can go ahead and credit any number of authors here,
% e.g. one 'row of three' or two rows (consisting of one row of three
% and a second row of one, two or three).
%
% The command \alignauthor (no curly braces needed) should
% precede each author name, affiliation/snail-mail address and
% e-mail address. Additionally, tag each line of
% affiliation/address with \affaddr, and tag the
% e-mail address with \email.
%
% 1st. author
\alignauthor
Parth Pratim Chatterjee\\
       \affaddr{Kalinga Institute of Industrial Technology}\\
       \email{parth27official@gmail.com}
\alignauthor
Thomas R. Henderson\\
       \affaddr{University of Washington}\\
       \email{tomhend@u.washington.edu}
}
% There's nothing stopping you putting the seventh, eighth, etc.
% author on the opening page (as the 'third row') but we ask,
% for aesthetic reasons that you place these 'additional authors'
% in the \additional authors block, viz.

% Just remember to make sure that the TOTAL number of authors
% is the number that will appear on the first page PLUS the
% number that will appear in the \additionalauthors section.

\maketitle
\begin{abstract}
%  * What was done? 
This paper reports on significant updates that were made to Direct Code Execution to adapt it to the evolution of application execution environments on Linux Systems 
and the standard libc library. An attempt to update the Linux Kernel backend which support the Linux network stack on custom simulation scripts through the 
net-next-nuse interface as a shared library, was made, implementation details and design decisions for which have been documented in this paper.
Finally, a docker based environment was worked on to ease the installation and usage of Direct Code Exection on Linux 
Systems, expanding the domain of Linux distributions supported by DCE also simplyfying the process of releasing new and maintaining old versions by the maintainers 
and upgrading for new users.
%  * Why do it? 
The newer Linux distributions introduced security features in it's standard system libc which restricted the major architecture and workflow of DCE. The network stack 
which was being used in DCE included a NUSE port on top of Linux Kernel 4.4.0, which has fallen out of date, so a later Linux kernel had to be ported into the 
net-next-nuse architecture.
%  * What were the results?
Direct Code Execution now supports execution on newer Ubuntu based distributions and all distributions which support Docker can now use DCE with minimal build steps 
and dependency installations. Linux Kernel 5.10.0 backed net-next-nuse-5.10 can now run Ipv4 applications on custom DCE simulation scripts.
\end{abstract}

% A category with the (minimum) three required fields
%\category{C.2.2}{ Computer-Communication Networks }{Network Protocols} [Protocol architecture]
%A category including the fourth, optional field follows...
%\category{I.6.5}{ Simulation and Modeling }{ Model Development}

%\terms{Theory}

\keywords{ns-3}

\section{Introduction}

\begin{itemize}
\item Introduce ns-3
\item Motivation of DCE: make complex network implementations available to ns-3
\item History of DCE, and challenge of maintaining it
\end{itemize}
% https://www.theregister.com/2020/01/06/linux_2020_kernel_systemd_code/
% estimated 27.8 million lines of code with 75,000-80,000 commits per year.

\begin{figure*}[h!]
  \centering
    \includegraphics[width=0.95\textwidth]{figs/architecture.png}

  \caption{DCE architecture (redrawn from Figure 1 of \cite{Tazaki13}) with recent extensions highlighted}
  \label{fig:architecture}
\end{figure*}

The rest of this paper is organized as follows: Section 2 provides an overview on the technical challenges due to Linux evolution.
Section 3 describes solutions.
Section 4 presents the results.  Section 5 related work.  Section 6 conclusions
and future work.

\section{Challenges}

\subsection{libio vtable mangling}
DCE requires to map FILE operations to it's internal custom handlers which work in synchronization with ns-3. DCE supports different operations on a FILE 
through a Unix file abstraction, which requires all FILE operation calls to be mapped through it. The requires us to modify the vtable entries for the FILE
object accordingly.

The vtable is a table maintaining references to functions called for virtual functions defined for a class or an entity. These functions can 
be overriden dynamically by user defined functions and the respective call for the corresponding virtual function in a derived class object can be bound 
to that function at runtime, unlike pre-defined functions which are static, fixed, and can be determined during compile time. The libc on Linux provides
this highly flexible feature for most user defined classes, but the case with the FILE structure is not the same.

The FILE structure is a library defined structure which defines the overall organization, orientation and properties of any file I/O stream opened
by the host application. It maintains different parameters for storing useful operational fields like the UNIX based file descriptor number of the 
opened stream, the read/write offsets and buffer addresses of the stream. The pseudoname for the FILE structure as seen inside libc is \textit{\_IO\_FILE}. 
Since, FILE is a library defined entity, the library provides its own set of implementation for all possible operations on an open FILE stream.
Whenever an \textit{\_IO\_FILE} stream is allocated by the kernel, a contiguous memory location is allocated as a block called \textit{\_IO\_FILE\_plus}. 
The \textit{\_IO\_FILE\_plus} structure looks like this.

\begin{lstlisting}[style=CStyle]     
struct _IO_FILE_plus
{
  FILE file;
  const struct _IO_jump_t *vtable;
};
\end{lstlisting}

By nature of the implementation of the kernel's memory allocation processes, the contents of a struct are allocated in contiguous memoory locations. 
This can be verified by the \textit{sizeof} operation of C showing that the sum of sizes of the individual members of a struct is equal to the size of
the struct object. Similarly, the FILE and the \textit{\_IO\_jump\_t} objects are allocated in contiguous memory locations. Specifically, the 
\textit{\_IO\_jump\_t} areas is interesting to DCE, as it defines the callbacks or reference pointers to the functions handlers for each supported file 
operation. Some of the callbacks which are interesting to DCE and its use cases are highlighed below.

\begin{multicols}{2}
\begin{itemize}
  \item \texttt{read}
  \item \texttt{close}
  \item \texttt{seek}
  \item \texttt{write}
  \item \texttt{stat}
\end{itemize}
\end{multicols}

This structure acts like the vtable for the FILE structure, but it does not behave like the oridinary vtable seen when working with virtual functions 
and derived classes, which are dynamic and supports run time bindings. This vtable is instead expected to behave as a statically bound vtable.  There exist 
other libc functions, like \textit{fopencookie}, to override some of the FILE operation implementations, but not all operations are supported, and 
\textit{fopencookie} also does not attach itself to a standard file I/O stream, 
instead working with user defined buffers a.k.a cookies). 

A key aspect of DCE is that it needs to \textit{hijack}
application operations like system calls, file I/O operations, and networking
system calls, and re-route them through 
corresponding handlers based on the application logic and simulation script implementation. Considering file I/O operations, DCE needs to have control 
over read/write/close/seek/stat operations of each open file, which requires
overwriting the vtable handlers with the corresponding handlers defined 
in DCE's stdio definition source files. 

Taking advantage of the contiguous memory allocation of the FILE and \textit{\_IO\_jump\_t}, one can overwrite the vtable with a custom vtable definition for all the operations needed for DCE. One can make a dummy \textit{\_IO\_FILE\_plus} 
pointer point to the explicitly casted FILE object. \textit{memcpy} the existing vtable to a local copy, modify and overwrite the stream operations with a 
custom written implementation, and then re-point the vtable field of the dummy \textit{\_IO\_FILE\_plus} to the local modified vtable.  This allows
control over stream operations, which can now be routed through and and to behave as ns-3 streams, Unix FD streams, etc. based on the type of file 
descriptor that is defined.  Although this control over FILE streams
to regulate stream buffer flushing and data redirection is productive for
DCE, these techniques can also be used maliciously for
\textit{buffer overflow attacks}, as it lets penetration 
testers to make use of tools like pwntools etc. to gain control over application execution and important run time CPU register values such as the 
\textit{rip} which is used for the relative addressing of application components(which is also how position-independent-executables work), which is more 
secure as compared to static addressing, where fixed address values of symbols and pointers could be gained by static analysis tools for run 
time application exploitations.

\sloppy Post libc-2.25, security features have been implemented to glibc to identify 
and block buffer overflow attacks on the FILE object. Whenever any FILE
operation is executed, glibc 
verifies if the FILE object's vtable can be trusted and is not corruputed or manipulated. To verify this, it makes a call to 
\textit{\_IO\_validate\_vtable}. Every libio vtable is defined in a unique section called \textit{libio\_IO\_vtables}. By definition, libc trusts
the vtable if the vtable of the current FILE object lies within this section. It checks if the offsets of this vtable lies between 
\textit{\_\_stop\_\_\_libc\_IO\_vtables} and \textit{\_\_start\_\_\_libc\_IO\_vtables}, if it does, we can continue with the operation, if not, libc 
conducts a final check by calling \textit{\_IO\_vtable\_check} which makes final checks on the FILE vtable pointer location, namespace and edge cases
where FILE * objects are passed to threads which are not in the currently linked executable.  When DCE overflows the \textit{\_IO\_FILE\_plus} and 
overwrites the \textit{\_IO\_jump\_t}, it does not lie in \textit{libio\_IO\_vtables} section and it also does not pass the pointer mangling sanity checks, leading to a \textit{\_\_libc\_fatal (\"Fatal error: glibc detected an invalid stdio handle\");}

\subsection{PIE loading and usage}
Position-independent-executables (PIE) are applications compiled with special compiler flags that allow the application to be loaded into an arbitrary
memory address, not depending on absolute symbol addresses. Since several DCE applications may need to be loaded into memory, and also have control over the
position of the \text{main} symbol of the loaded application, DCE needs to have position independent executables, so that when they are loaded into memory,
the symbol positions in memory are dynamic, giving control over when an application is launched in a simulation (which can be configured in the script 
using available ns-3 programming constructs for the \textit{DceApplicationManager} class). To implement this, DCE uses the \textit{CoojaLoader} which 
uses \textit{dlmopen} under the hood to load the executable into memory. In glibc library versions newer than 2.25, security checks have been introduced to identify such 
PIE objects being loaded through \textit{dlmopen}, and in case the \texttt{DF\_1\_PIE} flag is found in the object's ELF Dynamic headers, it will abort with an error.
 

\subsection{Linux Networking Stack for DCE}
DCE also provides implementation options for ns-3 simulations for the networking stack at the TCP/IP layers.  
Linux and FreeBSD stacks are available as alternatives to the native ns-3 TCP/IP implementation.
Script writers can configure the \textit{DceManager} class to use the chosen stack.
A Linux kernel library port allows DCE to export callback structures, abstracting the internal network data flow, coordinating Linux kernel synchronization, 
process creation, and DCE-kernel task, IRQ and tasklet scheduling and synchronization, to expose a Linux-like execution environment for host applications. 

Two potential projects or architectures to implement this are LKL and LibOS, which are discussed next.

\subsubsection{LKL}
The Linux Kernel Library (LKL) is a library port of the Linux kernel that, through some pre-shipped helper shell scripts, can be used to hijack
all system calls made by a host application and map them through the ported Linux kernel rather than through system defined implementations. It also allows one 
to setup network interfaces such as TAP, RAW, VDE , etc. with custom gateways, netmasks, IP addresses, etc. with the help of JSON configuration files. These helper
shell scripts make use of \texttt{LD\_PRELOAD} to reorder library loading to LKL written system calls to take control in place of the libc defined routines.
 
\subsubsection{LibOS}
LibOS uses the same internal architecture as does \textit{net-next-nuse}.
It is also a Linux kernel port that works on the principle of selective 
kernel module linking and patching. It defines special link time constructs
to include only specific kernel files and 
symbols that are needed for executing on top of the Linux kernel with \textit{nuse}, which works similar to LKL by hijacking system calls 
and rerouting them through nuse and kernel defined routines.

\subsubsection{LKL vs. LibOS}
We conducted a comparison of LKL and LibOS approaches according to different design parameters and considerations for a complex application framework such as
DCE with very specific demands from it's underying network stack.

\paragraph{Linux Kernel Support}
LKL, which was primarily designed to work as a Linux-as-userspace-library interface for applications to be dynamically linked to at runtime, is built on top of 
Linux 5.2. The kernel port design of LKL facilitates kernel version upgrades with little to no effort. Abstractly, the kernel upgrade process 
would permit a git rebase on the kernel version the user would want to use, and the project should compile with no major issues to deal with (some minor compiler, 
and Linux kernel header definition changes might come up, which should be resolvable with a bit of effort).

LibOS, which makes use of dynamic, selective Linux kernel linking, bridges the gap between application workspace and Linux kernel networking stack with
the help of glue code, kernel component connector code, and user application provided exported functions and callbacks for proper execution. This
architecture required LibOS to modify some of the internal Linux kernel files for additional components such as the slab memory allocator, which requires 
LibOS to setup preprocessor directives to select a slab allocator for specific Kconfig-defined compiler directives, to pass on control to LibOS routines 
whenever required. Currently, LibOS supports Linux kernel version 4.4.0. Upgrading to a newer Linux kernel version might be an intense process requiring one 
to deal with issues from header file changes, to complete changes in kernel components like the networking layer, memory maqnagement, namespace manager 
and kernel boot process.


\paragraph{Library kernel boot order and scheduling}
Both LKL and LibOS differ form each other on how it initializes the kernel and schedules tasks, which also determines many execution factors of DCE. 

Apart from certain functions that are initialized only in a user OS, such as hardware drivers and devices, network buses, NICs, etc., LKL spins up a high level CPU lock controlled thread, which makes calls to 
\texttt{start\_kernel}. Since LKL is a uniprocessor system, it initializes the kernel on a singular LKL thread, locks of 
which are synchronized with Linux scheduler calls, which are called when the scheduler decides to switch execution control 
to kernel level tasks for preemption, and other tasks, which require certain memory level moderation to achieve
atomic operation. Also, since the LKL CPU thread needs to be initialized before any Linux functions can be used by the 
application that is using the LKL library, it eventually disrupts the flow of DCE scripts, which work on a 
scheduling algorithm and specific ns-3 task context switch paradigms that are different from the Linux kernel,
making it wait for LKL's internal Linux kernel opertions to finish, before it can schedule other ns-3 lightweight
threads.  The net result of this tension is to cause simulation result to differ from real world observations.

LibOS does not depend on the actual \texttt{start\_kernel}. It makes 
use of its own \texttt{lib\_init} fuction which calls specific setups calls required by the network stack of the kernel to 
work, such as proc, VFS, ramfs, scheduler, etc. It also overrides scheduler member functions, syncing it with ns-3's 
scheduler. LibOS creates an ns-3 task copy for each kernel task which is created by the \texttt{copy\_process}, \texttt{create\_process},
etc. kernel functions. Each such task maps with itself a callback function which should be called once the wait time
for a task is over, or has been invoked as a part of a regular scheduling process. Once tasks are processed, they are 
also popped off the ns-3 task queue.

\section{Solutions}
\label{section:design}

\subsection{Custom glibc-2.31-based build}
The section discusses how the modified glibc is patched to disable problematic security checks which were discussed as some of the challenges previously
in the paper and also explains how the link procedure works.

The initial build step of DCE includes calling a script named \texttt{dcemakeversion.c}. This script is responsible 
for extracting the symbol table of the libc currently being linked to (all symbols for the libraries libc, libpthread, librt, lib and libdl). These 
libraries are the various modular extensions of the glibc providing features such as pthreads, math, dynamic library loading and the base libc 
library as well. The symbols are read from the Elf headers of the respective shared library .so file, and stored in a local .version file.
The symbol table for all of these libraries are important, as DCE generates a shared library called \texttt{libc-ns3.so} which is a 
for the local libc and DCE implementations, on top of which host applications are executed. This shared library defines 
all symbols which are defined in the local libc, as \texttt{NATIVE} and all the features implemented inside DCE as \texttt{DCE}. All other symbols that are new to 
DCE but not implemented by DCE, but that are a part of the local system libc, are then referenced in the .version files.  These symbols are then defined 
as well, to avoid any runtime symbol lookup errors. DCE then generates preprocessor mappings for all of the symbols. All DCE defined symbols will 
natively be mapped to DCE implemented versions of them, rather than to the system libc implementation, and all NATIVE defined symbols will be mapped 
to global namespace implementations, which are the ones already implemented in the system libc. Users would compile their applications on top of 
the system libc itself, but with an extra -fPIC and -pie flag, which allows us to load the applications dynamically into the DCE process address 
space. The next step is to load out the libc-ns3.so shared object file, and to call the \texttt{libc\_setup} function, which initializes the system call mappings. 
DCE also subsequently loads other libraries that have been generated in a similar way, such as \texttt{libpthread-ns3.so}, \texttt{librt-ns3.so}, \texttt{libm-ns3.so}, 
and \texttt{libdl-ns3.so}.As a final step, the host application is loaded and the main function of the application is called through the dlmopen object lookup constructs, 
which starts to now work on top of the custom libraries.

To override the vtable mangling security checks of glibc, it is necessary to override the security checks on vtable pointer mangling. 
The gcc compiler and linker options could be used to 
reconfigure the default build environment to build DCE on top of a customized
glibc. It is necessary to repoint the default libc directory to our custom glibc root. This is where the \texttt{--sysroot} option 
is used to set it to correct bake build directory. Following this, one can then add the custom glibc prioritized directory for library and header file lookup using the 
-L and -I options, respectively. Next, the rpath and rpath-link paths for ELF executables that could be linked to the shared objects at run time 
or link time, respectively, are set. Finally,the dynamic linker is set to the newly built library, using the -Wl--dynamic-linker flag.  All these changes
are placed under an unclosed -Wl --start-group, as DCE requires other linker flags, which can be added before inserting the ending -Wl --end-group.

\subsection{Bake Build Automation}

The process of the modified glibc and Linux compilation can be automated using Bake.
Bake has a source configuration option named \textit{patch}, which can safely apply a patch, without re-applying if it has already been applied before.  
This allows, for example, a large codebase to be fetched in its unaltered form, while Bake must maintain and apply only a small patch file to it.
Using this option, Bake applies the DCE-specific glibc patch, which disables the security checks on vtable mangling, and also disables the 
PIE object checks for dlmopen position independent executable loading. The glibc is then build using its standard build steps. The linux kernel headers
files are then installed into the \texttt{/usr} directory of a custom glibc's sysroot. This finishes with a standard Linux-like system root to use for building 
DCE without any issues. 


\subsection{Docker environnment for DCE}
Docker is one of the predominant tool used for software deployments considering the low host system dependence, flexibility to preconfigure execution 
applicaition environments and easy container, network and volume orchestration. DCE has a long dependency graph which includes dependencies across 
system libraries, other projects, Linux kernel and tools, and many other. Bake is a tool which can be used to download the external dependencies 
which should be built, but a lot of the system libraries and other host specific environment requirements needs to be assured by the end user, which
can be a trouble as some of the DCE release are supported only on specific Ubuntu machines with a specific libc version being used to avoid symbol 
missing errors during runtime. In such cases, a lot of the non supported Linux distribution users cannot use the project if they do not have access 
to a machine with supported specifications. 


\begin{figure}[!htb]
  \centering
    \includegraphics[width=0.37\textwidth]{figs/docker-architecture.jpg}

  \caption{DCE docker architecture composition}
  \label{fig:docker}
\end{figure}


The docker release consists of the above docker network setup which is orchestrated using a Docker Compose script. 
It consists of the docker container, environemnt variables and Volume. The docker 
container works on a custom image which uses ubuntu:20.04 as the base image. A docker image is collection of Image Layers. Each image layer is a 
Dockerfile command which helps create the final docker environemnt for that image. A layer can be a simple apt-get install command which installs 
system dependencies or libraries, or a local to docker file copy command. The docker image has been optimized to use minimal number of layers and 
all redundant layers are merged together. A layer for installing all system dependencies and libraries ensures all listed and unlisted bake 
dependencies for building DCE are satisfied. It also has a pre-compiled glibc bundled into the image itself. This helps us reduce the docker 
image size drastically as compared to the building from source everytime which might consume GBs, but this layer takes up 51.36 MB.
It also has a default environemnt variable 
\texttt{DCE\_WITH\_DOCKER=1} which is a flag variable for ns-3-dce waf script to try and execute a docker based build.

The environemnt variables which are fixed by the docker compose script are preset into the docker container when the service is started.
The container, as dicussed before, comes with a pre-compiled glibc-2.31 and so the user does not need to build the modified glibc again from source.
The environment variable \texttt{DCE\_WITH\_GLIBC} tells waf where to look for the modified glibc build directory. This environemnt variable can 
later be changed from inside the docker container to point to a custom glibc build directory if the user wishes to do so. This glibc build is then 
used by the ns-3-dce waf to compile ns-3-dce. The docker compose also sets up a volume. A docker volume is a mechanism which lets docker containers 
use persisting data from within the container. When a container is initally started without any attached containers, it is provided a temporary 
file space to work with, and any changes made during runtime by docker container is not persisted. Docker containers can attach volumes to themselves
and map specific directories from withing the docker container to the host directory. Any changes made to that directory withing the  docker container
is visible and persisted on the mapped host directory and also vice-vers. This is a key requirement for us, as ns-3 projects are used by both 
researchers and users who wish to just simulate scripts and also project maintainers, contributors and testers who need to be able to make changes 
to the ns-3 code locally using text editors and IDEs, and test it instantly, also access simulation output files on local file explorer and 
third party applications. So, we attach a volume which maps the /home/bake directory from within the docker container to the bake directory on 
the host.

The ns-3-dce waf is also edited accordingly. The waf script supports both docker based build and native builds on the same branch. It selects the glibc 
build directory based on waf configure and local environemnt variables and compiles ns-3-dce on top of it.

\subsection{net-next-nuse-5.10}
Net-next-nuse, introduced above, is an architecture port of the Linux kernel, which resorts to selective kernel module linking and which takes control over critical kernel 
components such as the slab allocator, task scheduler, workqueue-waitqueue handling, timer based function invocation, as well as some kernel 
utilities such as jiffies and random number generators. It also sets up emulations of the network system calls for both general socket networking 
operations as well netdevice based operations. All of this is exposed through an API initialised by a simulator initialisation call, \texttt{sim\_init}, 
which does a bidirectional mapping of the imported (DCE to Linux) and exported (Linux to DCE) function utilities.

\subsubsection{Background}
In this section, we will see what are some of the major challenges which net-next-nuse solves and the basics of how the Linux kernel is modified to 
accomodate this type of architecture.

Net-next-nuse understands that the way Linux Kernel behaves on the host, and the way the kernel interacts with the task 
scheduler and how the scheduler operates on the tasks, is completely different from how DCE manages fibers (called tasks in DCE terminology). 
DCE is based on context-based, lightweight threads called fibers, which require a custom scheduler, called the TaskManager in DCE. On an ordinary
host machine, when a user runs any application that spins up normal threads, , making the kernel responsible for jumping from one thread to another. 

Net-next-nuse uses fibres, which lets one jump from one thread to another only when one of the fibers yields to another fiber, and this context switch 
is done by a custom scheduler, which is written in DCE's TaskManager::Schedule(). This is beneficial because when using ordinary threads, 
the number of context switches the CPU has to do (push your thread data stack on to you address space, in and out), 
is a lot more expensive than making switches in fibers. This also eliminates the dependence of the kernel to bother about scheduling the threads on 
the different cores of the host CPU and maintaining mutex locks of the CPU inside the kernel library. To manage the isolated kernel threads inside the library , 
net-next-nuse  performs  a selective linking of 
kernel modules and utilities and to avoid linking the scheduler, workqueue, waitqueue and other such linked components, opting to instead implement
them using DCE imported functions mapped to TaskManager utilities. As a result, every time that some process wants to preempt and block, the kernel calls 
\texttt{schedule()} or \texttt{schedule\_timeout()}, and control is passed to DCE to take care of it. 

\subsubsection{Slab Allocator}
\sloppy When the kernel is booted up by net-next-nuse in \texttt{lib\_init(...)}, it calls some of the required initialization functions (initcalls). 
Most of these initcalls require the creation of a memory cache object called \texttt{kmem\_cache}. These memory slabs are used to allocate memory to child 
objects, using functions like \texttt{kmem\_cache\_alloc(...)} and functions as simple as \texttt{kmalloc(...)} to allocate space for a pointer object. The 
kernel already has slab allocators (two such allocators are SLUB and SLOB), but all of them allocate space internally, giving no control to DCE. We thus use a custom 
SLAB allocator called SLIB, which calls internal \texttt{DCE's malloc(...)} functions and sets up \texttt{kmem\_cache} data structures and constructors etc. It also 
implements compound and single page operations like\texttt{\_put\_page(...)}, \texttt{kfree(...)} and array space allocation. 

\subsubsection{Virtual Kernel Task and its Real Fiber Equivalent}
net-next-nuse uses exported functions and internal DCE calls to allow kernel task creation  which are exposes to net-next-nuse through an API during library object.
initialization.

\sloppy Every new task inside the kernel, for example, a syscall, 
creates an internal kernel thread.  When a system boots up, there has to be some initial task that invokes the \texttt{start\_kernel(...)} 
function. This is called the \texttt{init\_task}. Whenever a new kernel thread is to be created, a call to \texttt{kernel\_thread(...)} is made. Along with the 
function that is to be invoked in the thread, and the arguments to be passed, one must also pass along one more argument, known as clone flags. 
Internally, the kernel will try to clone a previous task as much as possible. These flags basically determine 
how far the kernel should go, as in cloning the structures. Some of the flags are \texttt{CLONE\_FS}, \texttt{CLONE\_VM}, \texttt{CLONE\_FILES}, etc., 
and \texttt{kernel\_clone(...)} passes a special clone flags data structure. This function further calls \texttt{copy\_process(...)}, which is responsible for checking 
which flags are enabled using a bitwise \texttt{\&} operator and calling the corresponding copy utility; for example, \texttt{copy\_fs(...)}, \texttt{copy\_files(...)}, 
and \texttt{copy\_cred(...)}.  It also makes a call to \texttt{dup\_task\_struct(...)}, which will bind changes of the current \texttt{task\_struct} to a new one and 
return it back. After receiving back a proper \texttt{task\_struct}, the \texttt{kernel\_clone(...)} schedules a fork for the newly created task.

\sloppy However, DCE is not a complete architecture-level port and thus lacks a 
clear definition of \texttt{init\_task}. So, for the very first \texttt{kernel\_thread}, the current task would be \texttt{init\_task}, 
so not having a proper definition of it creates a domino effect making all subsequent tasks have an incomplete structure, leading to possible segmentation faults.  
Furthermore, DCE does not use the kernel scheduler, but instead forks a kernel thread in such a way that the DCE scheduler can see it, and every time 
TaskManager::TaskWait(...)  is called on a particular task, DCE can put the current job to sleep, and give the other tasks/fibers a chance to 
execute their set of functions. 

\sloppy Therefore, to take  control over the kernel thread creation, we rewrite the \texttt{kernel\_thread(...)} function to call 
\texttt{lib\_task\_start(...)}, which calls the 
internal TaskManager::Start(...) to create a DCE task, fills up the task\_struct, sets the SimTask context with the task\_struct and returns back 
the process ID (pid) of the task. Also, \texttt{current}, as previously referred to above, is not a variable, but a macro, which calls \texttt{get\_current(...)}, which DCE hijacks and leads to TaskManager::TaskCurrent(...), which checks if the current task has a SimTask context. If it does, it returns it back. 
If it does not, it calls TaskManager::Start(...) to set it up.
 

\subsubsection{Scheduler Workqueue/Waitqueue Implementation}
DCE does not compile the 
original kernel implementation for the scheduler, but rather implements all of the key functions used by the networking stack. 
To understand the need to do this, lets imagine creating a socket, binding and listening to it and then making an \texttt{accept(...)} call, 
and the client application doesn’t seem to ever come up. The system would go to a standstill if it did not schedule the client application for it to issue a 
corresponding \texttt{connect(...)} call. This is one of the issues with LKL [4.2] as it is uniprocessor,
and is why we avoid linking those files, and 
rather pass the majority of the control to DCE and it's own scheduler, but without making a single change in DCE code, 
because DCE also has to work when simulation scripts use the default ns-3 network stack.  
DCE therefore rewrites all required Linux scheduler functions, which work based on LWP thread notifiers and are synchronized by 
the DCE Scheduler

\subsubsection{Kernel initialization}
The kernel initializers are divided into base init functions and roughly eight levels of initcalls, and export a start and end pointer of the list. 
Init functions are the ones directly called in the \texttt{start\_kernel(...)} or by certain device drivers, whereas initcalls are the ones which are stored 
in a special .initcall.init section of the final linked library. These are then copied to their respective positions through the linker.lds script.
In the \texttt{lib\_init} function, we call specific init functions. Some of the new additions made to initcall list are : 

\begin{itemize}
  \item vfs\_caches\_init\_early : To allow mounting of file systems such as \texttt{/proc} and file operations, which require the linked lists for 
  dcache(Directory Entry Cache, which helps fast lookup of recent paths) and inode.
  \item cred\_init : Steps up the \texttt{creds\_jar} cache which is required by the LSM \hyperref[Section_LSM]{3.4.7}, fs module and 
  while creds initialization for kernel task creation.
  \item proc\_root\_init : Sets up the \texttt{/proc} filesystem root cache, symlinks for mounts, net namespace for proc, and required directories.
\end{itemize}

The initcall invocation loop of lib\_init was also changed to use \texttt{initcall\_from\_entry} instead of a direct function call. 
This invocation function does a safe offset based function address lookup and calls the init method accordingly. 
Before we call all the init functions, I also initialized the \texttt{files} and \texttt{fs} cache pointers for cache based memory 
allocation using \texttt{kmem} in file operation methods, with slab objects aligned to a cache line, panic enabled on relocation errors, and also accounting enabled 
for the cache. These caches are required while setting up the \texttt{fs\_struct} and \texttt{files\_struct} for each new task in its 
corresponding \texttt{task\_struct}. Please see [5.9].

\subsubsection{Security LSM Module}\label{Section_LSM}
Certain kernel initialization procedures proceed only when the desired actions are allowed by the LSM. Below is a brief description of why the LSM is requied and how 
it is included into net-next-nuse.

In newer Linux releases \texttt{vfs\_kern\_mount(...)} validates the file system mount context and parses params using \texttt{security\_fs\_context\_parse\_param(...)},
which checks if the requested operation was blocked in the LSM tables. If this module is not enabled, it would return back ENOPARAM by default and fail. This makes it 
important for us to link the LSM, and setup our build/linking scipts accordingly.

The security module of the Linux kernel depends on a special configuration called the \texttt{CONFIG\_LSM}. The LSM module requires a few architecture
defined variables. Surprisingly, one cannot find them in the source code (.c or .h files) anywhere; rather, they are placed in a specific section
of the library/binary called .init.data section, through assembly code. net-next-nuse makes use of a linker script which can help us position variables inside
the library under specific sections. Below is the code for linker.lds to initialise the start and end of the LSM table during the link phase of the build.
\begin{lstlisting}[style=CStyle]
  . = ALIGN(CONSTANT (MAXPAGESIZE));
  .init.data : AT(ADDR(.init.data)) {    
		__start_lsm_info = .;	
		KEEP(*(.lsm_info.init))
		__end_lsm_info = .;    
		__start_early_lsm_info = .;	
		KEEP(*(.early_lsm_info.init))
		__end_early_lsm_info = .;
  }
\end{lstlisting}

\subsubsection{Kernel code and net-next-nuse alignment}
When net-next-nuse is loaded as a shared library into DCE and \texttt{lib\_init(...)} is called, it invokes certain kernel methods, which triggers 
kernel initialization routines which help setup critical components such as the mount namespaces, vfs, kernel threads and tasks, etc. 
Since Linux kernel 4.4.0, the
kernel has envolevd a lot in terms of how the kernel gets initialized. The section below discusses certain alignments which had to be made with respect to 
Linux kernel 5.10 to integrate net-next-nuse's architecture.

The \texttt{UCOUNT\_MNT\_NAMESPACES} of the current task’s nsproxy is used, which has an upper bound of \texttt{init\_task}‘s value.  Since
a complete architecture port is not needed, this value is not defined, and
would throw a segmentation fault, when \texttt{READ\_ONCE(...)} tries to make
an atomic operation on it to access it and increase the value by 1 if doing so doesn’t exceed the max limit, using the \texttt{atomic\_inc\_below(..)} macro.
We thus manually set the value for it, during task initialisation in \texttt{lib\_task\_start(...)}. The current task's nsproxy is used to allocate the 
\texttt{mnt\_namespace} which is required when VFS sets up the mount tree as a side effect of the initcall \texttt{mnt\_init}. We then set the value of the 
mount namespaces counter  limit to maximum possible mount namespaces.

\begin{lstlisting}[style=CStyle]
ucounts->ns->ucount_max[UCOUNT_MNT_NAMESPACES] = MNS_MAX;
\end{lstlisting}

\sloppy\texttt{sock\_init} is one of the many initcalls which are called when 
\texttt{sim\_init} is invoked by DCE. It is necessary to have initialised the proc file system completely,
so that the sysctl interface could be initialised by the \texttt{net\_sysctl\_init(...)}. We then register the sockfs filesystem using \texttt{register\_filesystem(...)} 
and go for a \texttt{kern\_mount(...)} which invokes the VFS kern mount function and as discussed above, it requires the \texttt{mnt\_init(...)} to correctly setup 
the mount environment. Initially, in net-next-nuse-4.4.0, \texttt{mnt\_init(...)}  was made a blank function.
In older kernel releases, sockfs file system init did not depend on a file system context, and required a mount function (\texttt{sockfs\_mount}) which
would directly mount the file system onto the kernel, but in commit [6.1], the socket file system mounting process was handed over to the
internal VFS mounting mechanism, thus breaking our setup. So, now we need to have the \texttt{mnt\_init(...)} and put in patches and glue codes wherever
needed.

\subsubsection{Native Kernel NetDevices}
Certain components of ns-3 require the setup of custom NetDevices which require a custom MTU and other flags, such as whether the device should enable multicast, 
or is it a point-to-point device, etc. These flags can be put together, and one can allocate a Linux netdevice struct for each such requirement using
the \texttt{alloc\_netdev(...)}, passing all of the configuration needed for details such as MTU, address length, and destructors, along with a \texttt{register\_netdev(...)} call.
This returns back a forward declared struct SimDevice which is used as a NetDevice object inside ns-3 which is required by ns-3's core networking modules.

\begin{itemize}
 \item Requests are provided in the form of API functions ...
 \item Notifications are provided as callback functions ...
\end{itemize}

\section{Performance Evaluation}
We describe some performance evaluations of the DCE code, following the
approach described in \cite{Tazaki13}.  The purpose of this section is to
compare the runtime performance of different versions of DCE and different
DCE configurations with the performance reported in \cite{Tazaki13} and with
a equivalent ns-3 (without DCE) simulation.

\begin{figure}[h!]
  \centering
    \includegraphics[width=0.47\textwidth]{figs/topology.png}
  \caption{Simulation topology (redrawn from Figure 2 of \cite{Tazaki13})}
  \label{fig:topology}
\end{figure}

The simulation topology used is shown in Figure \ref{fig:topology}, which
is duplicated from Figure 2 of \cite{Tazaki13}.  The topology is a simple
linear network with a variable number of hops, and a single UDP flow
is sent from client to server for the duration of the simulation.
The UDP data rate is configurable between 5 and 100 Mbps, and the UDP
payload size was a uniform 1470 bytes per packet.
To avoid per-hop congestion, link bandwidths are set to 1 Gbps.
For the purpose of runtime performance comparison, this topology is
suitable because it makes use of a DCE application (a custom UDP traffic
generator (\emph{udp-perf}) written in C and maintained as part of DCE),
and allows the enabling or disabling of kernel-mode operation.  Most
simulation operations involve the creation and transmission of packets.
The program to realize this topology is found as an example named
\emph{dce-udp-perf.cc} in the DCE codebase; we only had to make minor
updates to that program to realize the different configurations reported
herein, and we believe that this is the program originally used
in \cite{Tazaki13}.

We do not know what version of DCE was used in \cite{Tazaki13} but we
have reason to believe that it was not significantly different from the
most recent DCE 1.11 release of DCE (for Ubuntu 16.04), because there
has not been much development of DCE since the publication of \cite{Tazaki13}.
The new software we are reporting on herein (for Ubuntu 20.04, with newer
kernel code, and optionally with Docker) will be part of DCE 1.12 and
later releases.

ns-3 and DCE software can be compiled in debug or optimized mode.  For
these experiments, we compiled both ns-3 and DCE in optimized mode.  All
other software (e.g. net-next-nuse-4.4.0, glibc-2.31) was compiled in
the default mode specified for that software.

For the experiments, we used two different machines (for the two different
operating systems).  As DCE execution time is dominated by the single thread
CPU performance of a machine, execution of the same program on different CPUs
will lead to
different runtimes.  In \cite{Tazaki13}, an Intel Xeon 2.8 GHz machine was
used; this CPU is rated with a single thread performance of 384 by PassMark
Software's online CPU benchmark dataset\footnote{https://www.cpubenchmark.net}.
We used an Intel Core i7-4770 3.40 GHz machine for tests of DCE version 1.11
on Ubuntu 16.  This CPU has a single thread benchmark of 2167 according
to the same database.  For tests of DCE version 1.12 (Ubuntu 20.04), we used
a machine with an Intel Core i7-1065G7 1.30 GHz CPU.  This CPU has a
single thread benchmark of 2416 according to the same database.  Use of
different machines for the two different operating systems makes it difficult
to directly compare the DCE performance results, but because we are able to
run the same ns-3 simulation on both machines, we can calibrate the
performance by referencing DCE performance results in relation to the
corresponding ns-3 runtimes on each machine.

\begin{figure}[h!]
  \centering
    \includegraphics[width=0.47\textwidth]{figs/hops-vs-pps.png}
  \caption{Packet processing performance as a function of number of network hops}
  \label{fig:hops-vs-pps}
\end{figure}

Figure \ref{fig:hops-vs-pps} reproduces data from a similar figure (Figure 3)
in \cite{Tazaki13}, and provides some new data for current simulation runs.
In \cite{Tazaki13}, this figure was used to compare performance between
DCE and Mininet-HiFi.  Here, we compare the previous data on DCE from
\cite{Tazaki13} (read from the paper figure) with a similar experiment run
with DCE version 1.11 on Ubuntu 16.  A single simulation trial for different
values of network hops (4, 8, 16, 24, 48, 64) was run for 50 simulated
seconds, and the number of received packets was divided by the elapsed
wall clock time of each trial, to yield a packets-per-second measurement.
This experiment illustrates that the most recent release of DCE code has a
similar performance profile as that of \cite{Tazaki13}.  Because the
number of packets sent in each simulation is the same, the downward
slope of the curves as the number of hops increases indicates that larger
topologies (which cause more simulation events to be generated per packet)
will run more slowly.  We also note that an unexpected result of this
experiment was that the two curves would generally overlap, since the
simulations were conducted on machines with different CPU performance.
This could indicate a performance regression between the version of DCE
used in \cite{Tazaki13} and the most recent release; performance regression
has not been tracked as part of DCE maintenance.

\begin{figure}[h!]
  \centering
    \includegraphics[width=0.47\textwidth]{figs/rate-vs-time.png}
  \caption{Sending rate vs. wall clock time on a four hop topology for different DCE configurations (Ubuntu 16)}
  \label{fig:rate-vs-time}
\end{figure}

Figure \ref{fig:rate-vs-time} recreates a similar experiment 
(shown in Figure 5 of \cite{Tazaki13}) to highlight the impact of sending
rate on the wall clock time of the simulation trial, for different sending
rates on a four-hop topology.  As in \cite{Tazaki13}, a linear relationship
between the sending rate and wall clock time is expected, because the
number of hops per packet is held constant.  Again, we reproduce the
curve from \cite{Tazaki13} by reading their data from the paper figure.
There are several interesting observations.  First, we believe that the
closest comparison between the experiment from \cite{Tazaki13} is the
\emph{DCE 1.11 ELF} result.  This configuration used DCE with Linux kernel
version 4.4.0 and the ELF loader (known to lead to faster simulation
runs than the default Cooja loader).  Despite executing on a much faster
machine, the DCE 1.11 result is slower than the earlier reported simulation.
Again, we do not have specific information on how the previous run
was conducted (the website hosting the scripts is no longer active), and
it could possibly be due to performance regressions either with DCE
or with the Linux kernel library (\cite{Tazaki13} used a virtualized kernel
version of 2.6.36).  Aside from this difference, this figure can be used
to compare the execution time performance of different modes of DCE.
For comparison, we added a mode of operation to \emph{dce-udp-perf.cc} to
bypass the use of the \emph{udp-perf} DCE application and to use the
ns-3 UDP client application.  The blue line (bottom) is therefore a pure
ns-3 simulation, and runs fastest.  The next fastest execution is found
with \emph{DCE 1.11 ELF user}, which is using the ELF loader with DCE
running in user-mode, not kernel-mode.  This means that the DCE application
is using the ns-3 TCP/IP stack rather than the Linux kernel stack.  The
next fastest configuration is the \emph{DCE 1.11 ELF} configuration, using
DCE user-space and kernel code over the ns-3 simulated point-to-point links.
The final curve is the \emph{DCE 1.11} curve that uses the DCE user-space
and kernel code but with the default Cooja loader.  This curve shows that
the use of the ELF loader conveys a significant performance benefit.
While the Figure 5 of \cite{Tazaki13} includes plots for other network
diameters (up to 32 hops), we omit the larger topologies in our figure
for clarity; the larger topologies display similar linear relationships.
The horizontal dashed line depicted shows the performance point at which
the simulation time equals the wall clock time; only the DCE 1.11 curve
for data rates greater than 60 Mbps ran more slowly than real time in
these experiments.

\begin{figure}[h!]
  \centering
    \includegraphics[width=0.47\textwidth]{figs/rate-vs-time-dce12.png}
  \caption{Sending rate vs. wall clock time on a four hop topology for different DCE configurations (Ubuntu 20)}
  \label{fig:rate-vs-time-u20}
\end{figure}

Figure \ref{fig:rate-vs-time-u20} presents comparable simulation results
for the new version of DCE (the pending 1.12 and later releases).  As noted
above, the machine for these experiments had a faster CPU, and in comparing
the pure ns-3 results between both machines, the simulation runs on this
machine were about 25\% faster (although the ns-3 versions were different,
ns-3.34 on DCE 1.11 and ns-3.35 on DCE 1.12, this is not expected to
lead to any notable difference in performance).  The next curve above
the ns-3 curve, \emph{DCE-1.12 user}, can be compared with the
\emph{DCE-1.11 ELF user} in the previous figure.  In Ubuntu 20.04, the
ELF loader is not yet working, so the comparison is not exact, but the
experiment shows that despite the lack of ELF loader, use of the ns-3
stack in both cases leads to faster than real-time execution.

Next, compare the red curves, \emph{DCE-1.11} vs. \emph{DCE-1.12 kernel 4.4}.
Neither configuration used the ELF loader, and both used the same Linux
kernel library version (4.4).  However, the DCE-1.12 version uses the
modified glibc library.  The newer DCE version results in 75\% more runtime
despite being on the faster machine.  

The final two curves illustrate performance differences between the current
kernel library version 4.4 and a new kernel library (5.10), and the
difference between running DCE natively on the host operating system vs.
running within a Docker container.  The newer kernel library required
about 16\% more runtime.   Running within a Docker container was observed
to incur only about a 3\% performance hit.

As of this writing, the performance differences between the newer DCE 1.12
versions and the last released (DCE 1.11) versions have not been debugged
or profiled; the differences were discovered during the course of preparing
this paper.  The small performance cost of running within a Docker container,
however, may be worth the convenience to future users.

\section{Related Work}

\section{Conclusions}
%\end{document}  % This is where a 'short' article might terminate

\textcolor{red}{One conclusion to add is that it appears that performance regressions
can easily occur and that DCE should be regularly profiled.}

%ACKNOWLEDGMENTS are optional
\section{Acknowledgments}
This project was funded by the 2021 Google Summer of Code.
%
% The following two commands are all you need in the
% initial runs of your .tex file to
% produce the bibliography for the citations in your paper.
\bibliographystyle{abbrv}
\bibliography{revitalizing-dce}  % sigproc.bib is the name of the Bibliography in this case
% You must have a proper ".bib" file
%  and remember to run:
% latex bibtex latex latex
% to resolve all references
%
% ACM needs 'a single self-contained file'!
%
%APPENDICES are optional
%\balancecolumns
%\balancecolumns % GM June 2007
% That's all folks!
\end{document}
