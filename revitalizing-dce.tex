% This is "sig-alternate.tex" V2.0 May 2012
% This file should be compiled with V2.5 of "sig-alternate.cls" May 2012
%
% This example file demonstrates the use of the 'sig-alternate.cls'
% V2.5 LaTeX2e document class file. It is for those submitting
% articles to ACM Conference Proceedings WHO DO NOT WISH TO
% STRICTLY ADHERE TO THE SIGS (PUBS-BOARD-ENDORSED) STYLE.
% The 'sig-alternate.cls' file will produce a similar-looking,
% albeit, 'tighter' paper resulting in, invariably, fewer pages.
%
% ----------------------------------------------------------------------------------------------------------------
% This .tex file (and associated .cls V2.5) produces:
%       1) The Permission Statement
%       2) The Conference (location) Info information
%       3) The Copyright Line with ACM data
%       4) NO page numbers
%
% as against the acm_proc_article-sp.cls file which
% DOES NOT produce 1) thru' 3) above.
%
% Using 'sig-alternate.cls' you have control, however, from within
% the source .tex file, over both the CopyrightYear
% (defaulted to 200X) and the ACM Copyright Data
% (defaulted to X-XXXXX-XX-X/XX/XX).
% e.g.
% \CopyrightYear{2007} will cause 2007 to appear in the copyright line.
% \crdata{0-12345-67-8/90/12} will cause 0-12345-67-8/90/12 to appear in the copyright line.
%
% ---------------------------------------------------------------------------------------------------------------
% This .tex source is an example which *does* use
% the .bib file (from which the .bbl file % is produced).
% REMEMBER HOWEVER: After having produced the .bbl file,
% and prior to final submission, you *NEED* to 'insert'
% your .bbl file into your source .tex file so as to provide
% ONE 'self-contained' source file.
%
% ================= IF YOU HAVE QUESTIONS =======================
% Questions regarding the SIGS styles, SIGS policies and
% procedures, Conferences etc. should be sent to
% Adrienne Griscti (griscti@acm.org)
%
% Technical questions _only_ to
% Gerald Murray (murray@hq.acm.org)
% ===============================================================
%
% For tracking purposes - this is V2.0 - May 2012

\documentclass{sig-alternate}
\usepackage{array}

\usepackage{xcolor}
\usepackage{listings}

\lstdefinestyle{CStyle}{
    basicstyle=\footnotesize,
    breakatwhitespace=false,         
    breaklines=true,                 
    captionpos=b,                    
    keepspaces=true,                 
    numbers=left,                    
    numbersep=5pt,                  
    showspaces=false,                
    showstringspaces=false,
    showtabs=false,                  
    tabsize=2,
    language=C
}

\begin{document}
%
% --- Author Metadata here ---
\conferenceinfo{WOODSTOCK}{'97 El Paso, Texas USA}
%\CopyrightYear{2007} % Allows default copyright year (20XX) to be over-ridden - IF NEED BE.
%\crdata{0-12345-67-8/90/01}  % Allows default copyright data (0-89791-88-6/97/05) to be over-ridden - IF NEED BE.
% --- End of Author Metadata ---

\title{Revitalizing DCE}

\subtitle{[Extended Abstract]
\titlenote{A full version of this paper is available as
\textit{Author's Guide to Preparing ACM SIG Proceedings Using
\LaTeX$2_\epsilon$\ and BibTeX} at
\texttt{www.acm.org/eaddress.htm}}}
%
% You need the command \numberofauthors to handle the 'placement
% and alignment' of the authors beneath the title.
%
% For aesthetic reasons, we recommend 'three authors at a time'
% i.e. three 'name/affiliation blocks' be placed beneath the title.
%
% NOTE: You are NOT restricted in how many 'rows' of
% "name/affiliations" may appear. We just ask that you restrict
% the number of 'columns' to three.
%
% Because of the available 'opening page real-estate'
% we ask you to refrain from putting more than six authors
% (two rows with three columns) beneath the article title.
% More than six makes the first-page appear very cluttered indeed.
%
% Use the \alignauthor commands to handle the names
% and affiliations for an 'aesthetic maximum' of six authors.
% Add names, affiliations, addresses for
% the seventh etc. author(s) as the argument for the
% \additionalauthors command.
% These 'additional authors' will be output/set for you
% without further effort on your part as the last section in
% the body of your article BEFORE References or any Appendices.

\numberofauthors{1} %  in this sample file, there are a *total*
% of EIGHT authors. SIX appear on the 'first-page' (for formatting
% reasons) and the remaining two appear in the \additionalauthors section.
%
\author{
% You can go ahead and credit any number of authors here,
% e.g. one 'row of three' or two rows (consisting of one row of three
% and a second row of one, two or three).
%
% The command \alignauthor (no curly braces needed) should
% precede each author name, affiliation/snail-mail address and
% e-mail address. Additionally, tag each line of
% affiliation/address with \affaddr, and tag the
% e-mail address with \email.
%
% 1st. author
\alignauthor
AuthorName\\
       \affaddr{Institute for Clarity in Documentation}\\
       \email{mail@mail.com}
}
% There's nothing stopping you putting the seventh, eighth, etc.
% author on the opening page (as the 'third row') but we ask,
% for aesthetic reasons that you place these 'additional authors'
% in the \additional authors block, viz.

% Just remember to make sure that the TOTAL number of authors
% is the number that will appear on the first page PLUS the
% number that will appear in the \additionalauthors section.

\maketitle
\begin{abstract}
%  * What was done? 
This paper describes the design, implementation and validation
of the ns-3 model of the Licklider Transmission Protocol, the standard
transport protocol used to provide transmission reliability in Delay Tolerant Networks (DTNs).
%  * Why do it? 
DTNs are an emerging field whose principles are used to provide communications 
in extreme and performance-challenged environments, such as spacrecraft,underwater, or
disaster response scenarios. Evaluation of such environments requires the use of simulation tools.
As of now, there is a lack of precise simulation models of these protocols, and concretely within the ns-3 simulator.
%  * What were the results?
The ns-3 model presented in this paper accurately models the LTP protocol and offers ...
\end{abstract}

% A category with the (minimum) three required fields
\category{C.2.2}{ Computer-Communication Networks }{Network Protocols} [Protocol architecture]
%A category including the fourth, optional field follows...
\category{I.6.5}{ Simulation and Modeling }{ Model Development}

\terms{Theory}

\keywords{ACM proceedings, \LaTeX, text tagging}

\section{Introduction}

The rest of this paper is organized as follows: Section 2 provides an overview on Delay Tolerant
Networks and its transmission protocol standards. Sections 3 describes the design and implementation of the Licklider
Transmission Protocol ns-3 module. Section 4 presents the testing approach procedure. Section 5 shows
the validation procedure for achieving interoperability against existing implementations. Lastly, section 6 offers the conclusions and future work.


\section{Challenges}


\subsection{libio vtable mangling}

\subsection{PIE loading and usage}
Describe how reliability is achieved, introduce headers and types of segments (data, reports, etc.), distinction between red and green data.

\subsection{Linux Networking Stack for DCE (LKL vs LibOS) }
Talk about other DTN simulators/modules: the ONE \cite{one} , omnet++ \cite{omnet}, there are also some modules for ns-2 (not totally sure).

This subsection may be worth moving to the introduction as just a simple paragraph

\section{Solutions}
\label{section:design}

\subsection{Custom glibc-2.31 Based Build}

\subsection{Bake Build Automation}

\subsection{Docker environenment for DCE}

\subsection{net-next-nuse-5.10}

Communication between the LTP engine and the Client Service Instance can happen both ways. 
The client service makes requests to the LTP engine (start or cancel transmission) and 
the LTP engine issues back notifications (to report certain events or hand over received data). 
The design of these communications is a local implementation matter, in the ns-3 module:

\begin{itemize}
 \item Requests are provided in the form of API functions ...
 \item Notifications are provided as callback functions ...
\end{itemize}

The LTP may be run at different protocol layers in order to provide support for this
we provide Convergence Layer adapters ...

LTP uses engine IDs as its addressing system, we provide a lookup structure in the form of LTP to IP resolution tables ...


\section{Results}

\subsection{Docker vs Native DCE Simulation Tests}

\subsection{Performance : DCE vs. ns-3}

\subsection{Google BBR v1 Validation Results}

\section{Related Work}

\section{Conclusions}
%\end{document}  % This is where a 'short' article might terminate

%ACKNOWLEDGMENTS are optional
\section{Acknowledgments}
%
% The following two commands are all you need in the
% initial runs of your .tex file to
% produce the bibliography for the citations in your paper.
\bibliographystyle{abbrv}
\bibliography{ltp-ns-3-workshop}  % sigproc.bib is the name of the Bibliography in this case
% You must have a proper ".bib" file
%  and remember to run:
% latex bibtex latex latex
% to resolve all references
%
% ACM needs 'a single self-contained file'!
%
%APPENDICES are optional
%\balancecolumns
%\balancecolumns % GM June 2007
% That's all folks!
\end{document}
